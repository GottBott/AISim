% CISC475: Advanced Software Engineering
% CISC675: Software Engineering Principles & Practices
% Template for software development project report

\documentclass[11pt]{report}
\usepackage[nottoc,numbib]{tocbibind}
\usepackage{graphicx}
\usepackage[letterpaper,textheight=9in,left=1in,%
textwidth=6.5in,bottom=1in]{geometry}
\usepackage{hyperref}
  
\title{WebSys}

\author{%  
  Name of Chief Architect, Chief Architect\\
  Name of Manager, Manager\\
  Name of Documentarian, Documentarian\\
  \emph{all other members who contributed to this document}
  % note: the titles are optional; if the members of your team
  % did not use any titles, don't put them here
}

% I put this macro here to make it easy to write glossary entries...
\newlength{\glosslen}
\setlength{\glosslen}{\textwidth}
\addtolength{\glosslen}{-10mm}
\newcommand{\gloss}[2]{
  \vspace{3mm}

  \noindent\textbf{#1}\vspace{1mm}

  \hspace{5mm}\parbox{\glosslen}{#2}}

% Put your own macros here...


\begin{document}

\maketitle
\tableofcontents


\chapter{Preliminaries}

\section{Change Log}

Reverse-chronological list of changes to this document from
version 0.1 on.  

\section{Introduction}
Here we give a brief description of what our vision for the product
is and what problem it solves.

Say something about your team and the members.  (Did they have
different responsibilities?)

What is the URL for your team?  You must have a web presence in
today's competetive world.

Say something about your development process.  Did you use version
control?  Where, when, and how?  What are development
tools/methodologies did you use?  What about bug tracking?

Now say something about how this document is organized.

How was this document prepared?  (Answer: \LaTeX)


\chapter{Requirements}

You can structure this part how you want.   The structures here
are only suggestions.

\section{Stakeholders}

Who are the stake holders and how does each stakholder
depend on the outcome of this project?


\section{Problem Domain}
A requirements document typically begins with a description of the
problem domain.   Note that the components of this description
are \emph{not} requirements---they are ``givens'' describing
the world with which the software deals.

Here you usually define domain-specific vocabularly that will be used
throughout this report.  Be sure to give a precise definition of each
term before the first use of that term, and create an entry
for that term in the Glossary.

\section{Some Section}

The rest of the requirements section of the report must clearly
specify all requirements of the product.  The ingredients that
go into this description may include:
\begin{itemize}
\item clear, precise English prose
\item lists: each entry in the list describing one very specific
  requirement and giving a short name to that requirements;
  typically these lists are organized hierarchically
\item dataflow diagrams: together with descriptions in the prose
\item entity relationship diagrams: ditto
\item formal grammars
\item finite state automata
\end{itemize}

This would be a good place for a Use Case Context Diagram (Figure
\ref{fig:context}) together with text giving an overview of the use
cases and how they are related.

\begin{figure}
  \begin{center}
    This is my Use Case Context Diagram
  \end{center}
  \caption{Use Case Context Diagram}
  \label{fig:context}
\end{figure}

\section{Use Cases}

\subsection{Use Case 1: Goal 1}
\label{uc:sale}

This is excerpted from \cite{larman}.  I have reproduced it here to
show you a way to typeset a use case using \LaTeX.  Your use cases do
not have to contain as much details as these.  Use only fields that
are necessary and relevant to describe the behavior in each case.

\paragraph{Scope:} NetGen POS application

\paragraph{Level:} user goal

\paragraph{Primary Actor:} Cashier

\paragraph{Stakeholders and Interests:}
\begin{itemize}
\item Cashier: Wants accurate, fast entry, and no payment errors, as
  cash drawer shortages are deducted from his/her salary.
\item Salesperson: Wants sales commissions updated.
\item Customer: Wants purchase and fast service with minimal effort.
  Wants easily visible display of entered items and prices.  Wants
  proof of purchase to support returns.
\item Company: Wants to accurately record transactions and satisfy
  customer interests.  Wants to ensure that Payment Authorization
  Service payment receivables are recorded.  Wants some fault
  tolerance to allow sales capture even if server components (e.g.,
  remote credit validation) are unavailable.  Wants automatic and fast
  update of accounting and inventory.
\item Manager: Wants to be able to quickly perform override
  operations, and easily debug Cashier problems.
\item Government Tax Agencies: Want to collect tax from every sale.
  May be multiple agencies, such as national, state, and county.
\item Payment Authorization Service: Wants to receive digital
  authorization requests in the correct format and protocol.  Wants to
  accurately account for their payables to the store.
\end{itemize}

\paragraph{Preconditions:} Cashier is identified and authenticated.

\paragraph{Success Guarantee (or Postconditions):} Sale is saved.  Tax is
correcctly calculated.  Accounting and Inventory are updated.
Commissions recorded.  Receipt is generated.  Payment authorization
approvals are recorded.

\paragraph{Main Success Scenario (or Basic Flow):}
\begin{enumerate}
\item \label{uc:sale:arrive} Customer arrives at POS checkout with
  goods and/or services to purchase.
\item \label{uc:sale: start} Cashier starts a new sale.
\item \label{uc:sale:enter} Cashier enters item identifier.
\item \label{uc:sale:record} System records sale line item and
  presents item description, price, and running total.  Price
  calculated from a set of price rules.\newline \emph{Cashier repeats
    steps \ref{uc:sale:enter}--\ref{uc:sale:record} until indicates
    done.}
\item \label{uc:sale:total} System presents total with taxes
  calculated.
\item \label{uc:sale:ask} Cashier tells Customer the total, and asks
  for payment.
\item \label{uc:sale:pay} Customer pays and System handles payment.
\item System logs completed sale and sends sale and payment
  information to the external Accounting system (for accounting and
  commissions) and Inventory system (to update inventory).
\item System presents receipt.
\item Customer leaves with receipt and goods (if any).
\end{enumerate}

\paragraph{Extensions (or Alternative Flows):}

\begin{itemize}

\item[*a.] At any time, Manager requests an override operation:
  \begin{enumerate}
  \item System enters Manager-authorized mode.
  \item Manager or Cashier performs one Manger-mode operation. e.g.,
    cash balance change, resume a suspended sale on another register,
    void a sale, etc.
  \item System reverts to Cashier-authorized mode.
  \end{enumerate}

\item[\ref{uc:sale:arrive}a.] Customer or Manager indicate to resume a
  suspended sale.
  \begin{enumerate}
  \item Cashier performs resume operation, and enters the ID to
    retrieve the sale.
  \item System displays the state of the resumed sale, with subtotal.
    \begin{enumerate}
    \item Sale not found.
      \begin{enumerate}
      \item System signals error to the Cashier.
      \item Cashier probably starts new sale and re-enters all items.
      \end{enumerate}
    \end{enumerate}
  \item Cashier continues with sale (probably entering more items or
    handling payment).
  \end{enumerate}
\end{itemize}

\subsection{Use Case 2: Goal 2}

Here is the second use case.

\subsection{Use Case 3: Goal 3}

Here is the third.  And so on.

\section{Non-functional Requirements}

Other kinds of requirements: will it have to work on a network?  What
kind of platform can it be installed on?  Performance? Etc.?

\chapter{Design}

\section{Architecture}

How will the system be decomposed into (highest-level) modules?  Start
by listing the modules and giving each module a name.  For each
module, give a brief description of its responsibilities and a list of
the other modules it will use.

Now explain the rationalization for the design.  Why did you break it
down the way you did?  How does each module hide design decision(s)?
What alternative designs were considered, what were their
advantages/disadvantages, and why did you end up choosing this one.



\section{Detailed Design}

Now give a subsection on each module where you go into more detail.
First, what is the \emph{interface} for the module? This may be as
simple as a list of method signatures.

How will that module be decomposed into sub-modules?  What will the
uses relation look like on those submodules?  Perhaps UML class
diagrams would be appropriate.

\section{Traceability}

Ideally, you would now go through every requirement and explain
how it can be realized in the given design.  This may be too much,
so pick some interesting non-trivial subset of the requirements
and do it for them.

\chapter{Verification}

\section{Verification Strategy}

What is your overall verification strategy?   Note that you should
have thought about this before you started implementing the code.

\section{Testing 1....}

Now break down the description of the test suite in some way.
Perhaps into ``white box tests'' and ``black box tests''.

\section{Results}

How many tests are there, how many pass, how many failed, etc.
What interesting defects were discovered by the tests and 
how were they corrected (just some examples).

Give the results of coverage analyses as well.

\appendix

\chapter{Glossary}

It is a really good idea to have a glossary.  It contributes towards
internal completeness.

\gloss{modularity}{the quality of being divided into modules.  I will
  continue this definition just because I want to see what it looks
  like when it wraps around a line.  Not too shabby, if I do say so
  myself.}

\gloss{someword}{the definition of that word}

\gloss{someword2}{the definition of that word2}

\gloss{someword3}{the definition of that word3}

\gloss{someword4}{the definition of that word4}

\gloss{someword5}{the definition of that word5}

\begin{thebibliography}{9}

\bibitem{larman} Craig Larman, \emph{Applying UML and Patterns: An
    Introduction to Object-Oriented Analysis and Design and Iterative
    Development}, 3rd edition, Prentice Hall, Upper Saddle River, New
  Jersey, 2005.
  
\end{thebibliography}


\end{document}

